\documentclass[10pt,twocolumn,fleqn]{article}

%%\usepackage{a4wide}
%%\usepackage{amsthm}
\usepackage{times}
\usepackage{epsf}
\usepackage{graphicx}
\usepackage{epsfig}
\usepackage{graphics}
\usepackage{color}
\usepackage{url}
\begin{document}
\title{The MonetDB SQL front-end}
\author{ Niels Nes, Martin Kersten, {\small \textsc{CWI}, Netherlands}}
\date{}
\maketitle

\section{Introduction}
The language intergalactica for database applications is SQL. It evolved
through several phases of standardization to the version known as SQL-2003.
The SQL standard provides an ideal language framework, 
in terms of standardization committee viewpoints, 
which is hardly met by any of the existing (commercial) implementations.
This is largely due to legacy of old software and backward compatability
requirements from their client base.
 
The MonetDB database system is originally developed as a database backend
kernel with its own, low-level algebraic interface and scripting language.
The development of a SQL frontend has been purposely postponed to the point
where the kernel code base was sufficiently mature and field tested in
large, mission critial applications in the financial sector.

In 2002 the first version of the SQL front-end emerged. 
This late development made it possible to immediately start from the SQL'99
definition. The complete implementation of SQL'99 with supportive
tools for end-users to migrate applications from their current system
remains, however, a time-consuming task and conflicts with our primary tasks,
innovative research on database kernel technology.

This document has been written to give a quick introduction on the SQL
front-end, its limitations, and the way to use it. 
Information on the architecture of MonetDB and the available programming
language interfaces can be found on the website http://monetdb.cwi.nl.
A historical database is used in Section \ref{Tutorial}
set-up your first application.
A brief description of the application interfaces is given in Section \ref{Apis}.

\section{Functional overview}
\label{Functionality}
The MonetDB/SQL front-end is designed with the SQL99 standard as the starting
point. The initial goal was to provide Core SQL99 conformance and
selectively add feature packages.

%Packages included
The feature packages identified in the standard are often vendor specific
and their design reflect more the result of a democratic process then
the result of a scientific/engineering task.
Features may re-appear in multiple packages and vendors have a lot of
space to {\em enhance} the set.

The same holds for MonetDB, which is heavily based on an extensible 
infrastructure. Modules with novel functionality often grow out of
specific application needs. Once in a while some clean-up improves
the situation.
The feature package structure and its status is shortly described as follows
%check against standard for missing issues

\onecolumn
\begin{tabular}{l | l | l| l |}\\
ID	&Name		&Features		&MonetDB/SQL	\\\hline
PKGOO1	&Datetime	&Interval datatype		& Supported\\
	&		&Time zone specification	& \\
	&		&Full datetime			& \\
	&		&Interval qualifier 		& \\
PKG002	&Enhanced 	&Assertions		& Not supported\\
	&Integrity	&Referential delete actions	& \\
	&Management	&Referential update actions& \\
	&		&Constraint management		& \\
	&	 	&Subqueries in CHECK constraint	& \\
	&	 	&Triggers			& \\
	&	 	&FOR EACH STATEMENT triggers	& \\
	&	 	&Referential action RESTRICT	& \\
PKG003	&OLAP capabilities&CUBE and ROLLER		& \\
	& 		&INTERSECT operator		& \\
	&	 	&Row and table constructs	& \\
	&		&FULL OUTER JOIN		& \\
	&	 	&Scalar subquery values		& \\
PKG004	&SQL Persistent	&A programmatic extension to SQL that makes it	& Not supported \\
	&Stored		&suitable for developing more functionally	& \\
	&Modules(PST)	&complete applications	& Not supported\\
	&	 	&The commands CASE, IF, WHILE, REPEAT,	& \\
	&	 	&LOOP, and FOR	& \\
	&	 	&Stored Modules	& \\
	&	 	&Computational completeness	& \\
	&	 	&INFORMATION SCHEMA views	& \\
PKG005	&SQL Call-level	&SQL Call-level Interface support: an Application	& Not supported\\
&(CLI)	&Programming Interface (API) that enables SQL	& \\
	&Basic object support	&operations to he called that is very similar to the	& \\
	&	 & Open Database Connectivity (ODBC) standard	& \\
PKG006	&Basic object support&Overloading SQL-invoked functions & Not supported\\
	&		& and procedures	& \\
	&	 	&User-defined types with single inheritance: basic	& \\
	&	&SQL routines on user-defined types (including	& \\
	&	&dynamic dispatch)	& \\
	&	 	&Reference types	& \\
	&	 	&CREATE TABLE	& \\
	&	 	&Array support: basic array support, array	& \\
	&	&expressions, array locators, user-datatype (DDT )	& \\
	&	&array support, reference-type array support, SQL	& \\
	&	&routine on arrays	& \\
	&	 	&Attribute and field reference	& \\
	&	 	&Reference and deference operations	& \\
PKG007	& Enhanced object&ALTER TABLE, ADD	& \\
	& support	&Enhanced user-defined types (including	& \\
	&	&constructor options, attribute defaults, multiple	& \\
	&	&inheritance, and ordering clause)	& \\
	&	 	&SQL functions and type-name' resonation	& \\
	&	 	&Subtypes	& \\
	&	 	&ONLY in queries	& \\
	&	 	&Type predicate	& \\
	&	 	&Subtype treatment	& \\
	&	 	&User-defined CAST functions	& \\
	&	 	&DDT locators	& \\
	&	 	&SQL routines on user-defined types such as	& \\
	&	&identity functions and generalized expressions	& \\
\end{tabular}	


\subsection{SQL Data types}
The datatypes supported by MonetDB/SQL are summarized below.
The MonetDB kernel supports extensibility with new atomic types.
They are not immediately visible at the SQL layer. It requires
patching the catalog initialization routine of the parser
and possibly the semantic analyzer.

\begin{tabular}{l |l | l | l}
Category	& SQL03 & MonetDB/SQL & Comments\\
		& type	& type	&	\\\hline
binary		& BLOB	& blob	& 	\\
bit string	& bit	&	&	Not supported\\
		& bit varying&	&	Not supported\\
boolean		& boolean& boolean&		\\
character	& char	& char	&	\\
		& VARCHAR &	&	\\
		& national char(NCHAR)&& Not supported\\
		& NVARCHAR&	& 	Not supported\\
		& CLOB	&	&	Not supported\\
		& NCLOB	&	&	Not supported\\
numeric		& integer& int/integer &\\
		& smallint&smallint&	\\
		&numeric &numeric&	\\
		& decimal&decimal&	 \\
		& float & float&	\\
		& double precision& double&	\\
temporal	& date & date&	\\
		& time & time& SERVER CRASH	\\
		& time with time zone& ???\\
		& timestamp& timestamp&	\\
		& timestamp with time zone& Not supported\\
		& interval &	& Not supported\\\hline
{\em Additional}&	&	&	\\
		&	& oid	& unique identifier in database\\
		&	& url	& Not yet included\\
\end{tabular}

Types considered relevant by other vendors and not (yet) supported in the
current setup. We should identify the equivalent MonetDB/SQL types.

\begin{tabular}{l |l |l}
binary & SQLserver & up to 8K fixed-length binaries\\
bit	&  & single bit\\
nchar	&  & Unicode fixed length\\
ntext	&  & Unicode variable length\\
rowversion&  & unique version number for each row update\\
smalldatetime&  & limited time range\\
smallmoney &  & monetary values in limited range\\
sql\_variant&  & server supported types (e.g. 'ptr')\\
table &  & recursive (materialized) result\\
uniqueidentifier&  & oid over all servers\\
varbinary&  & variable bit string upto 8K\\
varchar(n)&  & text up to 8K\\\hline

set(v1..v64)& Mysql & limited bitset\\
year(2,4) & & small year representation\\\hline

bfile & Oracle & Blob stored outside database\\
raw & & bit string up to 2K\\
rowID & & ROWID pseudo column\\\hline

box	& Postgresql & 2D plane rectangle\\
cidr	& & IP-version\\
circle	& & circle in 2D plane\\
line	& & line in 2D plane\\
lseg	& & a collection of line segments in 2D\\
macaddr & & MAC address\\
money	& & US-style currency values\\
path	& & geometric path through 2D plane\\
point	& & in 2D\\
polygon & & closed collection of line segments\\
timespan & & \\
\end{tabular}

\subsection{Quick Reference}
The specifics of the MoneDB/SQL implementation are grossly organized
by category.

\subsubsection*{CASE}
The CASE function provides IF-THEN-ELSE functionality within
a SELECT or UPDATE statement. It is fully supported.

\subsubsection*{CAST}
The CAST operation  operation is SQL99 compliant.
The conversions are derived from the underlying storage engine facilities,
which embody all reasonable coercions between basic data types.

\subsubsection*{COMMIT TRANSACTION/ROLLBACK }
Transactions over both the catalog and database tables is supported.
The underlying store uses two facilities to ensure ACID properties.
First, a (binary) log table is maintained with all updates performed
on the database on behalf of a user. Second, base tables are assigned
a version number as part of the transaction initia

\subsubsection*{(DIS)CONNECT }

The CONNECT statement provides a mechanism to change the user role
for administrative actions or to overrule the default setting for
accessing the database server. [NOT SUPPORTED]

\subsubsection*{CREATE/DROP/USE DATABASE}
The databases are mapped to directories (maps) in a default location, as 
specified in the MonetDB initialization file. The current scheme is 
to simply allow any user to create a database by reserving a
directory to hold all information. The location of the database
log is similarly controlled.

The CREATE/DROP DATABASE are not implemented, because each one
should have its own instance of Mserver.

[stmt could be included. It seems that both instructions are
easy to implement and a rudimentary authorization scheme should be
applied, e.g. the user issuing the command is the owner of the directory
in which the database is placed]

\subsubsection*{CREATE/DROP FUNCTION}
Not yet implemented

\subsubsection*{CREATE/DROP INDEX}

The MonetDB kernel creates indices on the fly whenever profitable for
achieving good performance. Applying this statement is considered
an advice to create and maintain a hash index during a user session.

\subsubsection*{CREATE/DROP ROLE}
The SQL authorization scheme has been implemented.

\subsubsection*{CREATE/DROP SCHEMA}
The schema is a named collection if related objects under control of
a single authorization scheme. 

\subsubsection*{CREATE/DROP TABLE}
A table can be declared TEMPORARY ?
Create table supports NOT NULL, UNIQUE, REFERENCES..

\subsubsection*{CREATE/DROP VIEW}
Views definitions are separately administered.
[CASCADED|LOCAL] CHECK OPTION are supported?

Do we support updates through a view?
Do we keep track of the underlying related tables? e.g.
viewdep(viewname, view/table/...name)

\subsubsection*{DELETE}
\subsubsection*{GRANT/REVOKE}
The SQL authorization scheme has been implemented.

\subsubsection*{INSERT}
\subsubsection*{JOIN clause}
The LEFT/RIGHT OUTER joins are supported and also the NATURAL join

\subsubsection*{LIKE operator}
\subsubsection*{CONCAT operator}
\subsubsection*{Operators}
The common arithmetic operations are defined. A full list of the
available functions should be included as an appendix (see CREATE FUNCTION)

\subsubsection*{SAVEPOINT}
Savepoints within transactions can be set

\subsubsection*{SELECT }
\subsubsection*{SET CONNECTION}
\subsubsection*{SET ROLE }
\subsubsection*{SET TIMEZONE }
\subsubsection*{SET TRANSACTION}
\subsubsection*{START TRANSACTION}
\subsubsection*{TRUNCATE TABLE}
\subsubsection*{UPDATE}
\end{document}


\subsection{The todo-list}
SQL'99 and its successor SQL-2003 are extensive languages.
Several language constructs have been included/retained for
compatibility reasons, or a vendor pressing for it.
In the long run we intend to provide all functionality that
the standard pre-scribes. But, the road towards this holy grail
is long and should not be waited for.
The following table lists functionality that awaits an implementation
as of version MonetDB/SQL 1.0.16. It is more or less sorted by
our perception of need.

\begin{tabular}{l}
POSITION(str IN str)\\
SUBSTRING(str SIMILAR str ESCAPE chr)\\
OVERLAY function, i.e. substring replacment\
TRIM [BOTH|LEADING|TRAILING]( chr FROM str)\\
EXTRACT( datetimefile FROM interval)\\
WITH CHECK OPTION in view definitions\\
SYSTEM\_USER, CURRENT\_ROLE, CURRENT\_PATH\\
TIME ZONE manipulation and inspection (EXTRACT)\\
CAST (NULL AS <datatype>)\\
Column constraints: REFERENCES ARE [NOT] CHECKED [ON DELETE]\\
CREATE TEMPORARY TABLE which is autodropped \\
SQL functions \\
SQL client modules\\
OCTET\_LENGTH, BIT\_LENGTH, CARDINALITY functions\\
User defined types \\
Literal INTERVAL arithmetic, e.g. interval '6' day - interval '1' day\\
NATIONAL CHARACTER type and operators\\
CONVERT|TRANSLATE(str USING translation\_\\
DROP SCHEMA xx [CASCADE/RESTRICT]\\
CREATE LOCAL/GLOBAL TEMPORARY TABLE in procedures \\
CREATE DOMAIN\\
ARRAY type and operators\\
(Literal) ROW and reference type \\
Locator type for e.g. Blob\\
CONNECT statement to establish connections. \\
\end{tabular}
