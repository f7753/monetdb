\documentclass[10pt,twocolumn,fleqn]{article}

%%\usepackage{a4wide}
%%\usepackage{amsthm}
\usepackage{times}
\usepackage{epsf}
\usepackage{graphicx}
\usepackage{epsfig}
\usepackage{graphics}
\usepackage{color}
\usepackage{url}
\begin{document}
\title{The MonetDB SQL front-end}
\author{ Niels Nes, Martin Kersten, {\small \textsc{CWI}, Netherlands}}
\date{}
\maketitle

\section{Introduction}
The language intergalactica for database applications is SQL. It evolved
through several phases of standardization to the version known as SQL-2003.
The SQL standard provides an ideal language framework, 
in terms of standardization committee viewpoints, 
which is hardly met by any of the existing (commercial) implementations.
This is largely due to legacy of old software and backward compatability
requirements from their client base.
 
The MonetDB database system is originally developed as a database backend
kernel with its own, low-level algebraic interface and scripting language.
The development of a SQL frontend has been purposely postponed to the point
where the kernel code base was sufficiently mature and field tested in
large, mission critial applications in the financial sector.

In 2002 the first version of the SQL front-end emerged. 
This late development made it possible to immediately start from the SQL'99
definition. As soon as the SQL'03 spec became available, its content
was taken as the primary frame of reference.
The SQL development strategy is driven by immediate needs of the
user-base, such that less-frequently used features end up low on
the development stack.

This document has been written to give a quick introduction on the SQL
front-end, its limitations, and the way to use it. 
Proficiency in SQL is assumed.
Information on the architecture of MonetDB and the available programming
language interfaces can be found on the website http://monetdb.cwi.nl.
A historical database is used in Section \ref{Tutorial}
set-up your first application.
A brief description of the application interfaces is given in Section \ref{Apis}.

\section{Session overview}
discuss MapiClient and Mknife

\subsection*{Session parameters}
A limited number of SQL session parameters can be set with the
statement {\sc set} {\em variable = expression}.
The setting can be inspected with the {\sc sql} statement
{\sc select} * {\sc from} session, which currently includes
the following paramters.
\begin{tabular}{ll l}
auto\_commit & boolean & toggle the auto commit transaction mode\\
reply\_size & integer & limit the number of tuples returned by a query. (-1 implies everything being returned)\\
explain & "mil" | "sql" & toggle between {\sc sql} execution and
explanation of the intermediate code produced by the compiler\\
debug & integer value & toggle debugging flags inside the system. 
See for an overview the MonetDB configuration file\\
\end{tabular}

\section{Functional overview}
\label{Functionality}
The MonetDB/SQL front-end is designed with the SQL'03 standard as the reference
point. The primary goal is to provide Core SQL'03 conformance and
selectively add feature packages.

%Packages included
The feature packages identified in the standard are often vendor specific
and their design reflect more the result of a democratic process then
the result of a scientific/engineering task.
Features may re-appear in multiple packages and vendors have a lot of
space to {\em enhance} the set.

The same holds for MonetDB, which is heavily based on an extensible 
infrastructure. Modules with novel functionality often grow out of
specific application needs. Once in a while some clean-up improves
the situation.
The feature package structure and its status is shortly described as follows
%check against standard for missing issues
[TODO check the package structure against SQL'03]
Abbreviations: TBT (ToBeTested), FS  (Fully Supported), NS (Not Supported)
UD (Under Development)

\onecolumn
\begin{tabular}{l | l | l| l |}\\
ID	&Name		&Features		&MonetDB/SQL	\\\hline
PKGOO1	&Datetime	&Interval datatype		& TBT\\
	&		&Time zone specification	& N\\
	&		&Full datetime			& TBT\\
	&		&Interval qualifier 		& TBT\\\hline
PKG002	&Enhanced 	&Assertions		& NS supported\\
	&Integrity	&Referential delete actions	& NS\\
	&Management	&Referential update actions	& NS\\
	&		&Constraint management		& FS\\
	&	 	&Subqueries in CHECK constraint	& NS \\
	&	 	&Triggers			& NS\\
	&	 	&FOR EACH STATEMENT triggers	& NS\\
	&	 	&Referential action RESTRICT	& NS\\\hline
PKG003	&OLAP capabilities&CUBE and ROLLER		& NS\\
	& 		&INTERSECT operator		& TBT\\
	&	 	&Row and table constructs	& NS\\
	&		&FULL OUTER JOIN		& FS\\
	&	 	&Scalar subquery values		& FS\\\hline
PKG004	&SQL Persistent	&A programmatic extension to SQL that makes it	& UD \\
	&Stored		&suitable for developing more functionally	& \\
	&Modules(PST)	&complete applications	& Not supported\\
	&	 	&The commands CASE, IF, WHILE, REPEAT,	& \\
	&	 	&LOOP, and FOR	& \\
	&	 	&Stored Modules	& \\
	&	 	&Computational completeness	& \\
	&	 	&INFORMATION SCHEMA views	& \\\hline
PKG005	&SQL Call-level	&SQL Call-level Interface support: an Application	& FS\\
&(CLI)	&Programming Interface (API) that enables SQL	& \\
	&Basic object support	&operations to he called that is very similar to the	& \\
	&	 & Open Database Connectivity (ODBC) standard	& \\\hline
PKG006	&Basic object support&Overloading SQL-invoked functions & NS\\
	&		& and procedures	& \\
	&	 	&User-defined types with single inheritance: basic	& \\
	&	&SQL routines on user-defined types (including	& \\
	&	&dynamic dispatch)	& \\
	&	 	&Reference types	& \\
	&	 	&CREATE TABLE	& \\
	&	 	&Array support: basic array support, array	& \\
	&	&expressions, array locators, user-datatype (DDT )	& \\
	&	&array support, reference-type array support, SQL	& \\
	&	&routine on arrays	& \\
	&	 	&Attribute and field reference	& \\
	&	 	&Reference and deference operations	& \\\hline
PKG007	& Enhanced object&ALTER TABLE, ADD COLUMN	& TBT \\
	& support	&Enhanced user-defined types (including	& NS\\
	&	&constructor options, attribute defaults, multiple	& NS\\
	&	&inheritance, and ordering clause)	& NS \\
	&	 	&SQL functions and type-name' resolution	& NS \\
	&	 	&Subtables	& NS \\
	&	 	&ONLY in queries	& NS \\
	&	 	&Type predicate	& NS \\
	&	 	&Subtype treatment	& NS \\
	&	 	&User-defined CAST functions	& NS \\
	&	 	&UDT locators	& NS \\
	&	 	&SQL routines on user-defined types such as	& NS \\
	&	&identity functions and generalized expressions	& NS \\
\end{tabular}	


\subsection{SQL Data types}
The datatypes supported by MonetDB/SQL are summarized below.
The MonetDB kernel supports extensibility with new atomic types.
They are not immediately visible at the SQL layer. It requires
patching the catalog initialization routine of the parser
and possibly the semantic analyzer.

\begin{tabular}{l |l | l | l}
Category	& SQL03 & MonetDB/SQL & Comments\\
		& type	& type	&	\\\hline
binary		& BLOB	& blob	& FS 	\\
boolean		& boolean& boolean& FS		\\
character	& char	& char	& FS	\\
		& VARCHAR &	& FS	\\
		& national char(NCHAR)&& FS utf8 supported\\
		& NVARCHAR&	& 	NS\\
		& CLOB	&	&	NS\\
		& NCLOB	&	&	NS\\
numeric		& integer& int/integer & FS\\
		& smallint&smallint& FS	\\
		&numeric &numeric& FS	\\
		& decimal&decimal& FS	 \\
		& float & float& FS	\\
		& double precision& double& FS	\\
temporal	& date & date& FS	\\
		& time & time& FS \\
		& time with time zone& NS\\
		& timestamp& timestamp& FS	\\
		& timestamp with time zone& NS\\
		& interval &	& TBT\\\hline
multiset	& table & & NS \\
{\em Additional}&	&	&	\\
		&	& oid	& TBT\\
		&	& url	& TBT\\
GIS & box	& & NS\\
		&circle	& & NS\\
		&line	& & NS\\
		&lseg	& & NS\\
		&point	& & NS\\
		&polygon & & NS\\
\end{tabular}

\subsection{Quick Reference}
The specifics of the MoneDB/SQL implementation are grossly organized
by category.

\subsubsection*{CASE}
The CASE function provides IF-THEN-ELSE functionality within
a SELECT or UPDATE statement. The constructs COALISCE and IF\_NULL are
not supported.

\subsubsection*{CAST}
The CAST operation  are implented except for CAST( NULL as datatype).
The conversions are derived from the underlying storage engine facilities,
which embody all reasonable coercions between basic data types.

\subsubsection*{COMMIT TRANSACTION/ROLLBACK }
Only strict serializable transactions are supported.
Transactions over both the catalog and database tables is supported.
The underlying store uses two facilities to ensure ACID properties.
First, a (binary) log table is maintained with all updates performed
on the database on behalf of a user. Second, base tables are assigned
a version number as part of the transaction initia
Recovery is initiated upon server re-start.
Managing the log-space and backups cre urrently a manual activity.

\subsubsection*{(DIS)CONNECT }

The CONNECT statement provides a mechanism to change the user role
for administrative actions or to overrule the default setting for
accessing the database server. [NOT SUPPORTED]

\subsubsection*{CREATE/DROP/USE DATABASE}
The databases are mapped to directories (maps) in a default location, as 
specified in the MonetDB initialization file. The current scheme is 
to simply allow any user to create a database by reserving a
directory to hold all information. The location of the database
log is similarly controlled.

\subsubsection*{CREATE/DROP FUNCTION}
Not yet implemented

\subsubsection*{CREATE/DROP INDEX}

The MonetDB kernel creates indices on the fly whenever profitable for
achieving good performance. Applying this statement is considered
an advice to create and maintain a hash index during a user session.
Creation of an index has only effect when used in combination
with keys. Other use of indices is left to the underlying DBMS kernel.

\subsubsection*{CREATE/DROP ROLE}
The SQL authorization scheme has been fully implemented.

\subsubsection*{CREATE/DROP SCHEMA}
The schema is a named collection if related objects under control of
a single authorization scheme. 

\subsubsection*{CREATE/DROP TABLE}
A table can be declared TEMPORARY. The standard table construction
facilities are supported. The integrity constraint REFERENCES is implemented
with the RESTRICT clause only.

\subsubsection*{CREATE/DROP VIEW}
Views definitions fully supported with CHECK OPTION as being the default.
No update through views are currently allowed.

\subsubsection*{GRANT/REVOKE}
The SQL authorization scheme has been implemented.

\subsubsection*{JOIN clause}
The LEFT/RIGHT OUTER joins are supported and also the NATURAL join.

\subsubsection*{LIKE operator}
The pattern matching facilities for strings should be extended and tested.

\subsubsection*{CONCAT operator}
To be tested.

\subsubsection*{SAVEPOINT}
Savepoints are fully implemented.

\subsubsection*{SET TRANSACTION}
Strict serializability is the sole isolation level supported.

\subsubsection*{START TRANSACTION}
\subsubsection*{TRUNCATE TABLE}

\subsection{The todo-list}
SQL'99 and its successor SQL-2003 are extensive languages.
Several language constructs have been included/retained for
compatibility reasons, or a vendor pressing for it.
In the long run we intend to provide all functionality that
the standard pre-scribes. But, the road towards this holy grail
is long and should not be waited for.
The following table lists functionality that awaits an implementation
as of version MonetDB/SQL 1.0.16. It is more or less sorted by
our perception of need.

\begin{tabular}{l}
POSITION(str IN str)\\
SUBSTRING(str SIMILAR str ESCAPE chr)\\
OVERLAY function, i.e. substring replacment\
TRIM [BOTH|LEADING|TRAILING]( chr FROM str)\\
EXTRACT( datetimefile FROM interval)\\
WITH CHECK OPTION in view definitions\\
SYSTEM\_USER, CURRENT\_ROLE, CURRENT\_PATH\\
TIME ZONE manipulation and inspection (EXTRACT)\\
CAST (NULL AS <datatype>)\\
Column constraints: REFERENCES ARE [NOT] CHECKED [ON DELETE]\\
CREATE TEMPORARY TABLE which is autodropped \\
SQL functions \\
SQL client modules\\
OCTET\_LENGTH, BIT\_LENGTH, CARDINALITY functions\\
User defined types \\
Literal INTERVAL arithmetic, e.g. interval '6' day - interval '1' day\\
NATIONAL CHARACTER type and operators\\
CONVERT|TRANSLATE(str USING translation\_\\
DROP SCHEMA xx [CASCADE/RESTRICT]\\
CREATE LOCAL/GLOBAL TEMPORARY TABLE in procedures \\
CREATE DOMAIN\\
ARRAY type and operators\\
(Literal) ROW and reference type \\
Locator type for e.g. Blob\\
CONNECT statement to establish connections. \\
\end{tabular}

The short term priority list for development encompasses

\begin{tabular}{l}
SQL modules and external functions \\
GIS module re-vitalization\\
Inclusion of regexpr as a UDF \\
inclusion of the generated key functionality\\
ROLLUP, CUBE primitives\\
COPY into XML format\\
CORRESPONDING TO features\\
\end{tabular}
\end{document}
