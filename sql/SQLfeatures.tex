\documentclass[10pt,twocolumn,fleqn]{article}

%%\usepackage{a4wide}
%%\usepackage{amsthm}
\usepackage{times}
\usepackage{epsf}
\usepackage{graphicx}
\usepackage{epsfig}
\usepackage{graphics}
\usepackage{color}
\usepackage{url}
\begin{document}
\title{The MonetDB/SQL front-end}
\author{ Niels Nes, Fabian Groffen, Sjoerd Mullender, Martin Kersten \\{\small \textsc{CWI}, Netherlands}}
\date{}
\maketitle

\section{Introduction}
The language intergalactica for database applications is SQL. It evolved
through several phases of standardization to the version currently
known as SQL-2003.
The SQL standard provides an ideal language framework, 
in terms of standardization committee viewpoints.
It is, however,
hardly met by any of the existing (commercial) implementations.
This is largely due to legacy of old software and backward compatability
requirements from their client base.
 
The MonetDB database system is originally developed as a database backend
kernel with its own, low-level algebraic interface and scripting language.
The development of a SQL frontend has been purposely postponed to the point
where the kernel code base was sufficiently mature and field tested in
large, mission critial applications in the financial sector.

In 2002 the first version of the SQL front-end emerged. 
This late development made it possible to immediately start from the SQL'99
definition. As soon as the SQL'03 spec became available, its content
was taken as the primary frame of reference.
The SQL development strategy is driven by immediate needs of the
user-base, such that less-frequently used features end up low on
the development stack.

This document has been written to give a quick introduction on the SQL
front-end, its limitations, and the way to use it. 
Proficiency in elementary use of SQL is assumed.
Information on the architecture of MonetDB and the available programming
language interfaces can be found on the website http://monetdb.cwi.nl/.

\section{Getting started}
The MonetDB/SQL front-ends can be used in different ways. 
Simple textual interfaces are described in Section \ref{mapi} and
using them in through a web-form in Section\ref{form}.
The ODBC and JDBC application interfaces provide the bridges
to externally developed products. The interaction with
a generic DBMS IDE is described in Section \ref{ide}.
MonetDB comes with a home-brewed GUI, called Mknife, which
is shortly introduced in Section \ref{mknife}

The examples are illustrated using the VOC database, shortly introduced
next.
\subsection{The (Dutch) East Indian Company (VOC)}
Exploring the wealth of functionality offered by MonetDB is best
started using a toy database being distributed with all packages.
This database provides a peephole view into the administrative
system of a multi-national company, the {\em Verenigde Oostindische Companie (VOC)}.

The VOC was granted a monopoly on the trade in the East Indies on March 20, 1602
by the representatives of the provinces of the Dutch republic.
Attached to this monopoly was the duty to fight the enemies of the Republic and 
prevent other European nations to enter the East India trade.
During its history of 200 years, the VOC became the largest company of its kind, 
trading spices like nutmeg, cloves, cinnamon and pepper, and other consumer 
products like tea, silk and Chinese porcelain.
Her factories or trace centers were world famous: Desjima in Japan, 
Mokha in Yemen, Surat in Persia and of course Batavia, 
the Company's headquarters on Java.

The history of the VOC is an active area of research and
a focal point for multi-country heritage projects, e.g. 
TANAP\footnote{http://www.tanap.net/content/about/heritage.htm},
which includes a short historic overview of the VOC written by
world expert on the topic F. Gaastra.
The archives of the VOC are spread around the world, but a large
contingent still resides in the 
National Archive
\footnote{ http://www.nationaalarchief.nl/ }
, The Hague.

The archives comprise over 25 million historical records.
Much of which has not (yet) been digitized.

The MonetDB tutorial is based on the material published in the book 
J.R. Bruijn, F.S. Gaastra and I. Schaar, <i> Dutch-Asiatic Shipping in the 17th and 18th Centuries</i>, which
gives an account of the trips made to the East and ships returned safely
(or wrecked on the way). A total of 8000 records are provided. 
They include information about ship name and type, captain,
the arrival/departure of harbors along the route, personnel accounts, 
and anecdotal information.

\subsection{The VOC schema}
The VOC database contains several tables introduced briefly.
\footnote{The database load script can be downloaded from http://monetdb.cwi.nl/Assets/voc\_dump.sql}
An introduction by the historians is included in the Appendix.

{\footnotesize
\begin{tabular}{l l}
sys.voyages	& the trips reported\\
sys.craftsmen	& the craftsmen on board \\
sys.soldiers	& the soldiers on board \\
sys.seafearers	& the seafearers on board \\
sys.impotenten	& the passengers on board \\
sys.passengers	& the passengers on board \\
sys.total	& the tally of all persons on board\\
sys.invoices	& the trading records \\
\end{tabular}
}

The central table is called {\em voyages}. It contains
a condense description of the boat and its visits to
harbors on the route. 
For many trips there exist
detailed log books, part of this information is
included to illustrate the harseness of the sea (wrecking).

{\footnotesize
\begin{tabular}{l l l}
number	& integer(9)	& trip key\\
number\_sup	& char(1)	& trip key\\
trip	& integer(9)	&\\
trip\_sup	& char(1)	&\\
boatname	& varchar(50)	& as registered\\
master	& varchar(50)	& captain's name\\
tonnage	& integer(9)	& tonnage in metric tons\\
type\_of\_boat	& varchar(30)	& kind of ship\\
built	& varchar(15)	& year of construction\\
bought	& varchar(15)	& proprietary ship\\
hired	& varchar(15)	& leased from owner\\
yard	& char(1)	& where it was built\\
chamber	& char(1)	& the investment company\\
departure\_date	& varchar(15)	&\\
departure\_harbour	& varchar(30)	&\\
cape\_arrival	& varchar(15)	&\\
cape\_departure	& varchar(15)	&\\
arrival\_date	& varchar(15)	&\\
arrival\_harbour	& varchar(30)	&\\
next\_voyage	& integer(9)	& return trip\\
particulars	& varchar(530)	& life and dead on board\\
\end{tabular}
}

The craftsmen, soldiers, seafearers, impotenten, and passenger table 
all have the same attribute structure. It starts with the foreign key
into the voyages table. 
Impotenten were people seeking fortune in Asia, but who were
unable to work for the VOC.
The remaining fields denote their fate, boarding the craft,
leaving at the Cape, or died on the way.
disembarkmen

{\footnotesize
\begin{tabular}{l l l}
number	& integer(9)	& \\
number\_sup	& varchar(1)	& \\
trip	& integer(9)	& \\
trip\_sup	& varchar(1)	& \\
one	& integer(9)	&  \\
two	& integer(9)	& \\
three	& integer(9)	& \\
four	& integer(9)	& \\
five	& integer(9)	& \\
\end{tabular}
}

At the end of a trip the cargo was sold and the proceeds handed over
to the investment company.  Occassionaly the cargo was sold
directly for another chamber. The value is denoted 17th century
Dutch guilders, a strong international currency.
{\footnotesize
\begin{tabular}{l l l}
number	& integer(9)	& \\
number\_sup	& varchar(1)	& \\
trip	& integer(9)	& \\
trip\_sup	& varchar(1)	& \\
invoice	& integer(9)	& \\
chamber	& varchar(3)
\end{tabular}
}

The SQL script does not include any suggestions for indices.
[should we]
\subsection{Loading the database}
\label{mapi}
In this section we illustrate the textual interface to MonetDB/SQL using the
MapiClient program. This combination has been the baseline for all textual
interactions with a MonetDB server.
Failure to get it working often implies a wrong or inconsistent installation.

To start a session, you should have installed both the
MonetDB back-end and SQL front-end according to the 'HowToStart' descriptions.
\footnote{http://monetdb.cwi.nl/GetGoing/Download/index.html}
The next step is to start the MonetDB server.

The MonetDB server does not contain hardwired knowledge on its SQL frontend.
It must be initialized and told to accept SQL sessions. In a separate window
start the server to access the VOC database:

{\footnotesize
\begin{verbatim}
$ Mserver --dbname=VOC 
# Monet Database Server V4.3.19
# Copyright (c) 1993-2006, CWI. All rights reserved.
# Compiled for i686-pc-linux-gnu/32bit; dynamically linked.
# Visit http://monetdb.cwi.nl/ for further information.
monet>module(sql_server);
monet>
\end{verbatim}
}

The server is initialized by loading the SQL module. It will listen
to a default port for requests.
A failure to load the module indicates an improperly installed SQL
frontend. See the installation documentation for further information and
tests.

If you get an error message, e.g. 'port already in use', you might have 
an Mserver process hanging around which should be terminated first.

The distribution comes with a simple textual interface with
command-line history features, called MapiClient.
To connect to the running MonetDB/SQL server using the
default user and authorization information, type:

\begin{verbatim}
$ MapiClient  -lsql 
sql>
\end{verbatim}
To be assured of a proper SQL environment, you could inspect the catalog
tables as follows, the actual output may slightly differ in your version:
\begin{verbatim}
sql> select name from tables;
\#-
\# 3 \# querytype
\# tables \# table_name
\# name \# name
\# varchar \# type
\# 11 \# length
\# 17 \# tuplecount
\# 0 \# id
[ "modules"     ]
[ "schemas"     ]
[ "tables"      ]
[ "columns"     ]
[ "keys"        ]
[ "idxs"        ]
[ "keycolumns"  ]
[ "types"       ]
[ "tmp_tables"  ]
[ "tmp_columns" ]
[ "users"       ]
[ "user_role"   ]
[ "auths"       ]
[ "privileges"  ]
[ "history"     ]
[ "sessions"    ]
[ "env" ]
\end{verbatim}

If you do not see the sql$>$ prompt, or worse get an error
message 'connection refused' you may call for an expert.
It indicates use of an inconsistent MapiClient and Mserver,
or basic problems in setting up a TCP/IP connection.

The MapiClient provides a simple textual interface including
a command line history. A synopsis of the commands can be obtained
by typing '?'.

The next step is to initialize and load the VOC database, if such has
not been done before by your MonetDB system administrator. It involves
creation of the SQL table, loading the tuples, and committing the
transaction. All this is packaged in an SQL script file you should
be able to find in the distribution tree under the name voc\_dump.sql.
At the MapiClient console give the command to read ca 30,000
sql statements  (3.6 Mb) from the designated file.

\begin{verbatim}
sql> <voc_dump.sql
sql>
\end{verbatim}

The VOC database in its full incarnation contains over 8000
trip reports. To retrieve part of it, you have to use selection and order
by predicates. A few illustrative SQL statements and snippets of
the answers are shown.
{\footnotesize
\begin{verbatim}
sql> select boatname,destination_arrival 
sql> from voyage 
sql> order by destination_arrival;
# boatname destination_arrival # name
# varchar date # type
# 5 # tuplecount
[ "LES DEUX SOEURS",           nil ]
[ "BRESLAU",            1783-08-06 ]
[ "POTSDAM",            1783-08-09 ]
[ "MAGDEBURG",          1783-09-05 ]
[ "GERECHTIGHEID",      1783-11-08 ]
sql> create view R as 
sql> select boatname as boot, tonnage as weight 
sql> from voyage;
sql> select * from R;
# boot weight # name
# varchar int # type
# 5 # tuplecount
[ "BRESLAU",            1150]
[ "LES DEUX SOEURS",     582]
[ "POTSDAM",            1150]
[ "GERECHTIGHEID",      1150]
[ "MAGDEBURG",          nil]
sql> select boatname 
sql> from voyage
sql> where tonnage between 0 and 1000;
# boatname # name
# varchar # type
# 1 # tuplecount
[ "LES DEUX SOEURS" ]
\end{verbatim}
}

\subsection{Simple web-form}
\label{form}
\subsection{A DBMS gui}
\label{ide}
\subsection{Mknife}
\label{mknife}
Mknife is a 'Swiss army knife' GUI to ease the interaction with a running 
MonetDB server. It provides a convenient setting to developed, test and 
tune queries. Although it is possible to compose rather complex applications,
including form-based data entry, its strength lies in supporting pre-cooked 
query scenarios. This requirement stems from our research activities in managing 
multimedia information in a database context. A setting where the database 
is a priori filled with a large collection of reference material, 
e.g. images or video keyframes, and automated tools enrich the database 
with derived features. Mknife is used here to ease experimentation and 
occasionally demonstrate insightful results.

Mknife comes along with a few examples based on the VOC
data. The screenshot below illustrates what you may expect.
For more details, we refer to the Mknife manual, hosted on
the
MonetDB website \footnote{http://monetdb.cwi.nl/GetGoing/Usage/Mknife/index.html}.

\begin{figure*}
\begin{center}
%include a picture
%\includegraphics[height=6cm,width=8cm]{mknife.jpg}
\caption{Mknife in action }
\end{center}
\end{figure*}


\section{Functional overview}
\label{Functionality}
The MonetDB/SQL front-end is designed with the SQL'03 standard as the reference
point. The primary goal is to provide Core SQL'03 conformance and
selectively add feature packages.

\subsection*{Session parameters}
A limited number of SQL session parameters can be set with the
statement {\sc set} {\em variable = expression}.
The setting can be inspected with the {\sc sql} statement
{\sc select} * {\sc from} env, which currently includes
the following parameters.
\begin{figure*}
\begin{tabular}{ll l}
auto\_commit & boolean & toggle auto commit transaction mode\\
reply\_size & integer & limit tuples returned by a query.\\
	&& (-1 implies everything )\\
explain & 'plan' & show MIL query plan\\
	& 'history' & keep a statement log \\
debug & integer & toggle debugging flags inside the system.\\
 && See for an overview the MonetDB configuration file\\
\end{tabular}
\caption{SQL session variables}
\end{figure*}

%Packages included
\subsection*{Packages}
The feature packages identified in the standard are often vendor specific
and their design reflect more the result of a democratic process then
the result of a scientific/engineering task.
Features may re-appear in multiple packages and vendors have a lot of
space to {\em enhance} the set.

The same holds for MonetDB, which is heavily based on an extensible 
infrastructure. Modules with novel functionality often grow out of
specific application needs. Once in a while some clean-up improves
the situation.
The feature package structure and its status is shortly described as follows
%check against standard for missing issues
[TODO check the package structure against SQL'03]
Abbreviations: TBT (ToBeTested), FS  (Fully Supported), NS (Not Supported)
UD (Under Development)

\onecolumn
\begin{tabular}{l | l | l| l |}\\
ID	&Name		&Features		&MonetDB/SQL	\\\hline
PKGOO1	&Datetime	&Interval datatype		& TBT\\
	&		&Time zone specification	& N\\
	&		&Full datetime			& TBT\\
	&		&Interval qualifier 		& TBT\\\hline
PKG002	&Enhanced 	&Assertions		& NS supported\\
	&Integrity	&Referential delete actions	& NS\\
	&Management	&Referential update actions	& NS\\
	&		&Constraint management		& FS\\
	&	 	&Subqueries in CHECK constraint	& NS \\
	&	 	&Triggers			& NS\\
	&	 	&FOR EACH STATEMENT triggers	& NS\\
	&	 	&Referential action RESTRICT	& NS\\\hline
PKG003	&OLAP capabilities&CUBE and ROLLER		& NS\\
	& 		&INTERSECT operator		& TBT\\
	&	 	&Row and table constructs	& NS\\
	&		&FULL OUTER JOIN		& FS\\
	&	 	&Scalar subquery values		& FS\\\hline
PKG004	&SQL Persistent	&A programmatic extension to SQL that makes it	& UD \\
	&Stored		&suitable for developing more functionally	& \\
	&Modules(PST)	&complete applications	& Not supported\\
	&	 	&The commands CASE, IF, WHILE, REPEAT,	& \\
	&	 	&LOOP, and FOR	& \\
	&	 	&Stored Modules	& \\
	&	 	&Computational completeness	& \\
	&	 	&INFORMATION SCHEMA views	& \\\hline
PKG005	&SQL Call-level	&SQL Call-level Interface support: an Application	& FS\\
&(CLI)	&Programming Interface (API) that enables SQL	& \\
	&Basic object support	&operations to he called that is very similar to the	& \\
	&	 & Open Database Connectivity (ODBC) standard	& \\\hline
PKG006	&Basic object support&Overloading SQL-invoked functions & NS\\
	&		& and procedures	& \\
	&	 	&User-defined types with single inheritance: basic	& \\
	&	&SQL routines on user-defined types (including	& \\
	&	&dynamic dispatch)	& \\
	&	 	&Reference types	& \\
	&	 	&CREATE TABLE	& \\
	&	 	&Array support: basic array support, array	& \\
	&	&expressions, array locators, user-datatype (DDT )	& \\
	&	&array support, reference-type array support, SQL	& \\
	&	&routine on arrays	& \\
	&	 	&Attribute and field reference	& \\
	&	 	&Reference and deference operations	& \\\hline
PKG007	& Enhanced object&ALTER TABLE, ADD COLUMN	& TBT \\
	& support	&Enhanced user-defined types (including	& NS\\
	&	&constructor options, attribute defaults, multiple	& NS\\
	&	&inheritance, and ordering clause)	& NS \\
	&	 	&SQL functions and type-name' resolution	& NS \\
	&	 	&Subtables	& NS \\
	&	 	&ONLY in queries	& NS \\
	&	 	&Type predicate	& NS \\
	&	 	&Subtype treatment	& NS \\
	&	 	&User-defined CAST functions	& NS \\
	&	 	&UDT locators	& NS \\
	&	 	&SQL routines on user-defined types such as	& NS \\
	&	&identity functions and generalized expressions	& NS \\
\end{tabular}	


\subsection{SQL Data types}
The datatypes supported by MonetDB/SQL are summarized below.
The MonetDB kernel supports extensibility with new atomic types.
They are not immediately visible at the SQL layer. It requires
patching the catalog initialization routine of the parser
and possibly the semantic analyzer.

\begin{tabular}{l |l | l | l}
Category	& SQL03 & MonetDB/SQL & Comments\\
		& type	& type	&	\\\hline
binary		& BLOB	& blob	& FS 	\\
boolean		& boolean& boolean& FS		\\
character	& char	& char	& FS	\\
		& VARCHAR &	& FS	\\
		& national char(NCHAR)&& FS utf8 supported\\
		& NVARCHAR&	& 	NS\\
		& CLOB	&	&	NS\\
		& NCLOB	&	&	NS\\
numeric		& integer& int/integer & FS\\
		& smallint&smallint& FS	\\
		&numeric &numeric& FS	\\
		& decimal&decimal& FS	 \\
		& float & float& FS	\\
		& double precision& double& FS	\\
temporal	& date & date& FS	\\
		& time & time& FS \\
		& time with time zone& NS\\
		& timestamp& timestamp& FS	\\
		& timestamp with time zone& NS\\
		& interval &	& TBT\\\hline
multiset	& table & & NS \\
{\em Additional}&	&	&	\\
		&	& oid	& TBT\\
		&	& url	& TBT\\
GIS & box	& & NS\\
		&circle	& & NS\\
		&line	& & NS\\
		&lseg	& & NS\\
		&point	& & NS\\
		&polygon & & NS\\
\end{tabular}

\subsection{Quick Reference}
The specifics of the MoneDB/SQL implementation are grossly organized
by category.

\subsubsection*{CASE}
The CASE function provides IF-THEN-ELSE functionality within
a SELECT or UPDATE statement. The constructs COALISCE and IF\_NULL are
not supported.

\subsubsection*{CAST}
The CAST operation  are implented except for CAST( NULL as datatype).
The conversions are derived from the underlying storage engine facilities,
which embody all reasonable coercions between basic data types.

\subsubsection*{COMMIT TRANSACTION/ROLLBACK }
Only strict serializable transactions are supported.
Transactions over both the catalog and database tables is supported.
The underlying store uses two facilities to ensure ACID properties.
First, a (binary) log table is maintained with all updates performed
on the database on behalf of a user. Second, base tables are assigned
a version number as part of the transaction initia
Recovery is initiated upon server re-start.
Managing the log-space and backups cre urrently a manual activity.

\subsubsection*{(DIS)CONNECT }

The CONNECT statement provides a mechanism to change the user role
for administrative actions or to overrule the default setting for
accessing the database server. [NOT SUPPORTED]

\subsubsection*{CREATE/DROP/USE DATABASE}
The databases are mapped to directories (maps) in a default location, as 
specified in the MonetDB initialization file. The current scheme is 
to simply allow any user to create a database by reserving a
directory to hold all information. The location of the database
log is similarly controlled.

\subsubsection*{CREATE/DROP FUNCTION}
Not yet implemented

\subsubsection*{CREATE/DROP INDEX}

The MonetDB kernel creates indices on the fly whenever profitable for
achieving good performance. Applying this statement is considered
an advice to create and maintain a hash index during a user session.
Creation of an index has only effect when used in combination
with keys. Other use of indices is left to the underlying DBMS kernel.

\subsubsection*{CREATE/DROP ROLE}
The SQL authorization scheme has been fully implemented.

\subsubsection*{CREATE/DROP SCHEMA}
The schema is a named collection if related objects under control of
a single authorization scheme. 

\subsubsection*{CREATE/DROP TABLE}
A table can be declared TEMPORARY. The standard table construction
facilities are supported. The integrity constraint REFERENCES is implemented
with the RESTRICT clause only.

\subsubsection*{CREATE/DROP VIEW}
Views definitions fully supported with CHECK OPTION as being the default.
No update through views are currently allowed.

\subsubsection*{GRANT/REVOKE}
The SQL authorization scheme has been implemented.

\subsubsection*{JOIN clause}
The LEFT/RIGHT OUTER joins are supported and also the NATURAL join.

\subsubsection*{LIKE operator}
The pattern matching facilities for strings should be extended and tested.

\subsubsection*{CONCAT operator}
To be tested.

\subsubsection*{SAVEPOINT}
Savepoints are fully implemented.

\subsubsection*{SET TRANSACTION}
Strict serializability is the sole isolation level supported.

\subsubsection*{START TRANSACTION}
\subsubsection*{TRUNCATE TABLE}

\section{Configuration files}
\label{configfile}

\section{The SQL todo-list}
SQL'99 and its successor SQL-2003 are extensive languages.
Several language constructs have been included/retained for
compatibility reasons, or a vendor pressing for it.
In the long run we intend to provide all functionality that
the standard pre-scribes. But, the road towards this holy grail
is long and should not be waited for.
The following table lists functionality that awaits an implementation
as of version MonetDB/SQL 1.0.16. It is more or less sorted by
our perception of need.

\begin{tabular}{l}
POSITION(str IN str)\\
SUBSTRING(str SIMILAR str ESCAPE chr)\\
OVERLAY function, i.e. substring replacment\
TRIM [BOTH|LEADING|TRAILING]( chr FROM str)\\
EXTRACT( datetimefile FROM interval)\\
WITH CHECK OPTION in view definitions\\
SYSTEM\_USER, CURRENT\_ROLE, CURRENT\_PATH\\
TIME ZONE manipulation and inspection (EXTRACT)\\
CAST (NULL AS <datatype>)\\
Column constraints: REFERENCES ARE [NOT] CHECKED [ON DELETE]\\
CREATE TEMPORARY TABLE which is autodropped \\
SQL functions \\
SQL client modules\\
OCTET\_LENGTH, BIT\_LENGTH, CARDINALITY functions\\
User defined types \\
Literal INTERVAL arithmetic, e.g. interval '6' day - interval '1' day\\
NATIONAL CHARACTER type and operators\\
CONVERT|TRANSLATE(str USING translation\_\\
DROP SCHEMA xx [CASCADE/RESTRICT]\\
CREATE LOCAL/GLOBAL TEMPORARY TABLE in procedures \\
CREATE DOMAIN\\
ARRAY type and operators\\
(Literal) ROW and reference type \\
Locator type for e.g. Blob\\
CONNECT statement to establish connections. \\
\end{tabular}

The short term priority list for development encompasses

\begin{tabular}{l}
SQL modules and external functions \\
GIS module re-vitalization\\
Inclusion of regexpr as a UDF \\
inclusion of the generated key functionality\\
ROLLUP, CUBE primitives\\
COPY into XML format\\
CORRESPONDING TO features\\
\end{tabular}
\newpage
\subsection*{Appendix: }
The introduction given below is an OCR version from the book:
J.R. Bruijn, F.S. Gaastra and I. Schaar, {\em Dutch-Asiatic Shipping in the 17th and 18th Centuries}.
\subsection*{Introduction}
This book presents tables which give a virtually complete survey of the 
direct shiping between the Netherlands and Asia between 1595-1795.
This period contains, first, the voyages of the so-called Voorcompagnieen and,
then, those for and under control of the Verenigde Oostindische Compagnie (VOC).
The survey ends in 1795.
That year saw an end of the regular sailings of the VOC between the Netherlands
and Asia, since, following the Batavian revolution in January, the Netherlands
beame involved in war with England. 
The last outward voyage left on 26 December 1794. 
After news of the changed situation in the Netherlands was received in Asia,
the last homeward voyage took place in the spring of 1795. 
The VOC itself was disbanded in 1798.

In total 66 voyages of the voorcompagnieen are listed, one more than 
the traditionally accepted number. The reconnaissance ship, POSTILJON, 
from the fleet of Mahu and De Cordes, that was collected en route 
is given its own number (0022). Since the attempt of the 
Australische Compagnie to circumvent the monopoly of the VOC can be 
considered as a continuation of the voorcompagnieen the voyage of 
Schouten and Le Maire is also listed (0196-0197).

For the rest, exclusively the outward and homeward voyages of the VOC 
are mentioned in the tables. Of those there were in total 4722 outward 
and 3359 homeward. The administration of the company was strictly followed, 
so that, for example, the voyage of Hudson in 1609 (0133) is listed,
but not that of Roggeveen in 1721-1722. 
Voyages of East Indiamen that were driven off course, and arrived for 
instance in Su­rinam, or those which went no further than the Cape are 
mentioned, as opposed to those of warships of the five Admiralties which, 
from 1783, were sent to Asia to protect the fleets and possessions of the VOC.

The sources of the journeys consist primarily of the archives of the VOC in 
the Algemeen Rijksarchief in The Hague. They are, on the one hand,
the so-called `Uitloopboeken' and ship registers, and, on the other, 
the `Overgekomen Brieven en Papieren' (OBP's). The latter contain the 
regular reports on the arrival and departure of ships in Batavia and other 
Asiatic harbours. In addition, the 'Overgekomen Brieven van de Kaap de 
Goede Hoop' and some other, more dispersed sources must be mentioned
The data on the voyages of the voorcompagnieen derive above all from 
sources published by the Linschoten Vereeniging.

In volume I, the principal sources are described extensively and the origin 
of the information on each voyage is given. In addition, that volume 
contains an introduction on the organisation of the VOC's shipping, 
which also includes an analysis and summary of the data presented in the tables.
Various other supplementary information, such as the value of the export 
from the Netherlands, only available by year, is also published there.

The tables follow closely the material presented in the major sources 
(`Uitloopboeken' en OBP's). Since these sources are not uniform over a 
period of almost two centuries, the level of completeness of the 
information given for each voyage also varies.

\subsubsection*{Homeward voyage.}
During the compilation of the tables it became necessary in a few cases 
to add an A to some numbers. This occurred 5 times, in the following places:
5022, 5980, 5987, 6246 and 6649. 
Similarly in three cases a number had to be left open. 
The following numbers have not been used: 4605, 5027 and 8215.

The voyage number is followed by a figure which shows whether the ship is 
making its first, second or subsequent -voyage. The outward and homeward 
voyages are counted separately. The first voyage from the Netherlands and 
the subsequent homeward voyages are both shown by a `1'.
Occasionally a ship was built or acquired in Asia. The first outward voyage 
of such a ship is considered as its second voyage. 

\subsubsection{Ship's name.}
A uniform spelling has been chosen for the numerous variants given in the 
sources. In alphabetical ordering and in the index, the most relevant word 
was chosen. Thus the WAPEN VAN, HOORN (0243) is given under HOORN, 
the HOF NIET ALTIJD ZOMER (2380) under ZOMER and the 
VROUWE REBECCA JACOBA (3668) under REBECCA. It should be noted that especially 
in the seventeenth century ships' names were frequently provided 
with additions which were not used in a consistent fashion. 
The AMSTERDAM (0431) was sometimes called NIEUWAMSTERDAM, the WITTE OLIFANT 
(0533), the OLIFANT. The most frequent name is given in the tables. 
In the eighteenth century, especially, ships' names were frequently changed, 
or they used each other's names. This is always mentioned under the heading 
Particulars and in the index.

\subsubsection*{Master's name}
Similarly, a uniform spelling has been chosen for the name of the master, 
generally schipper, but in the eighteenth century also a kapitein or 
kapiteinluitenant. The index is arranged by surname or patronymic. 
The death of a master, if known, during the voyage is shown by at.

\subsubsection*{Tonnage}
The volume of the ships is given in metric tons. The sources give the 
figures in lasten (1 last = 2 tons). After 1636, however, information 
in fasten is no longer of any value, as, for fiscal reasons, the VOC's 
figures were kept artificially low. From then on the volume has been 
calculated on the basis of the measurements of the ships, according to a 
simple formula (volume in lasten = length x breadth x depth in Amsterdam 
feet, divided by 200; 1 ft. = 28,3 cm). The results of this calculation 
have been doubled and are given in the tables. This method and the 
problems regarding the assessment of the ships' volume is described in 
Volume I. In a number of cases where inconsistent information was found, 
both calculations are given, thus e.g. 600/850. 

\subsubsection*{Type of ship}
Occasionally, in those cases where this is mentioned in the sources, the 
type of the ship is given in the same column as the tonnage. In general, 
the most frequent type of ship, the retourschip (East Indiaman) is 
not mentioned in the sources. Therefore, where the type of ship is 
not mentioned, it may often be assumed that an East Indiaman is meant.
The various other types - hoeker, kat, pinas, jacht, fluit, paketboot - 
are given in Dutch.

\subsubsection*{Built}
The year given in this column refers to the year in which the ship was built.
If the ship was hired or bought by the VOC, then this is mentioned in the
column, together with the year in which the transaction occurred.

\subsubsection*{Yard}

The place is given where the ship was built. The chambers of the VOC had 
their own yard. `A' refers to Amsterdam, `Z' to Zeeland, `D' to Delft, `R' 
to Rotterdam, `H' to Hoorn, and `E' to Enkhuizen. When a ship was hired 
or bought by the VOC, the letter indicates the chamber that was responsible 
for the transaction.

The ships of the voorcompagnieen did not belong to a chamber. In these 
cases, `A' indicates that a ship was built at an Amsterdam yard. 
The chambers also had no part in the buying or building of ships by 
the Hoge Regering in Batavia. In these cases the place of building 
or purchase in Asia is given.

\subsubsection*{Chamber}
With the outward voyages, this column gives the chamber which equipped the ship;
with the homeward, the chamber to which the ship was addressed.
There is no entry in this column for ships organised by the voorcompagnieen.

\subsubsection*{Departure}
Under this heading is given the date and place of departure from Europe, 
Asia or the Cape of Good Hope. A date like 03-02-1645 refers to 3 February 1645.

Where sailings from the Republic are concerned, the date given refers to 
the departure from the roads. Amsterdam, Hoorn and Enkhuizen ships generally 
left from Texel roads, Zeeland ships from the Wielingen or the 
roads of Rammekens, and Rotterdam and Delft ships from Goeree. 
Sometimes, ships were forced by storms or damage to return to the roads 
for a time or they sought shelter in one of the estuaries on the coast 
of Holland or Zeeland. Where possible, this is mentioned under the 
heading Particulars. In general the first date of departure is given in 
the tables, but in some cases, a later date has been chosen, in 
deference to the sources.

As for leaving Batavia, departure from the roads of the town was decisive, 
and not, as is frequently described in the Company papers, the reaching of 
the `open sea' after passing the Sunda Strait. Where departures from other 
Asian ports are concerned in general only the Company establishment 
from which the ship sailed is given. Thus Ceylon is mentioned in the columns, 
but it can be assumed that most ships left from the Bay of Galle, at 
the southern point of the island. China is given for ships which left from 
Canton, and the date refers to departure from the roads at Whampoa. 
Bengal is given for ships which left from the anchorage in the Ganges 
close to the VOC-establishment at Hughly.

\subsubsection*{Call at the Cape}
The data in this column give the arrival at (above) and the departure (below) 
from the Cape of Good Hope. In general no distinction is made between Table 
Bay and False Bay. Mention is made, when given in the sources, of ships 
which put in to the more northerly Saldanha- and St. Helena Bays. When a 
ship sailed past the Cape, this is denoted by `no call'. When it is not 
known whether the ship stopped at the Cape at all - especially frequent before 
the foundation of the refreshment station there in 1652 - the column is 
left blank.

\subsubsection*{Arrival}
The third column contains the date and place of arrival in Asia, Europe or, 
when that was the destination of the journey, at the Cape of Good Hope. 
The place of arrival is given in the same manner as that of departure, though,
in addition to the estuaries mentioned above, ships sometimes arrived in the 
Netherlands via the Vlie or at Delfzijl. The place of arrival in Asia 
refers to the establishment reached, unless the sources specify the actual port.

\subsubsection*{On Board}
It is possible to differentiate the number of those on board into various 
categories. For the outward journey, these are seaf(arers), sold(iers), 
crafts(men), and pass(engers). The craftsmen are those who were employed to 
perform some particular service in Asia, and are thus not part of the crew 
as such. `Passengers' is in fact a residual category, including high officials 
of the Company, including ministers of religion with their wives and servants, 
but also slaves and stowaways. Whenever such a differentiation is not possible,
which is especially the case in the early years, a figure for the total is 
given. Italics are used for this, or when the figures refer to more than 
one category. Only those categories are mentioned which were on board.
Therefore, when one category is mentioned, this implies that the others 
were not represented on board.

The sources for the return voyages are of a different kind and normally far 
less complete. They are totally absent for the journey between the Cape and 
the Netherlands. However, another category must be mentioned, namely 
the impotenten, who for various reasons were released from active service 
for the VOC and sent back to Europe. With regard to many voyages the sources 
only give the number of passengers and impotenten, and not the number of 
sailors and soldiers. Obviously, the absence of figures under these 
headings does not imply that there were none on board.

Information on the outward voyages is divided into six columns:\\
I. The number on board at departure.\\
II. The number dying between the Netherlands and the Cape. Frequently this
figure refers to all the categories together, even when the other information
is available per category. In such cases this figure is printed in italics.
III. The number who leave the ship at the Cape.\\
IV. The number who come on board at the Cape.\\
V. The number dying on the whole voyage. Subtraction of III from V gives the
number dying between the Cape and Asia.\\
VI. The number on board on arrival in Asia. Three columns are given for the homeward voyages:

I. The number on board at departure.\\
II. The number dying en route to the Cape.\\
III. The number who went from board at the Cape.\\

The figures in the various columns are taken from different sources which 
are not always consistent with each other. Therefore the figures on changes in 
the number of those on board during the voyages do not always tally with 
those on the size of the crew at departure and arrival.

\subsubsection*{Invoice value}
For the return voyages, the total value of the ship's cargo, according to the 
invoice made up in Batavia or some other establishment, is given, as is the 
chamber for which the cargo was destined. Generally, this was for the chamber 
under whose jurisdiction the ship sailed, but occasionally a proportion of 
the cargo was for one or more of the other chambers.

\subsubsection*{Particulars}
Under the last heading details deriving from the basic sources are given. 
They are generally incidental and as such not to be placed in one of the 
preceding columns. Because the sources are not the same across the whole 
period, and at times less complete, the extent and sort of material under 
this heading could not be consistent.

In so far as it is available, information deals with the ports of call on 
the journey, with the details of changes in the composition of the crew and 
with the eventual fate of the ship. For the return voyage, the name of the 
fleet-commander is generally given, and, after his name, the number of the 
ship he was on. Finally, where neces­sary, differences in data between 
arious sources are indicated. Occasionally, particulars from a published 
ource are added.

\subsubsection*{Corresponding number}
This number, placed at the far right of the tables, denotes the next homeward 
voyage of the ship in volume II (naturally absent when the ship remains in 
Asia), or, in volume III, for homeward voyages, the number of the 
ship's previous outward voyage. In those cases where the ship was aquired in 
Asia, no corresponding number is given for the first homeward voyage from Asia.

Due to the long duration of the preparation of these two volumes there are 
some inconsistencies in the text of the particulars and in the use of language.
\end{document}
